\documentclass[informe.tex]{subfiles}
\begin{document}
  
  \section{Opciones de uso}

    Ambos problemas fueron resueltos usando Matlab. Para ejecutarlos describimos a continuación las opciones posibles.
  
    \subsection{Reducción de dimensionalidad}
      Para la ejecución del código hay dos archivos .m, uno de ellos el que utilizamos para generar todos los resultados de manera autom\'atica: main.m el cual varía las repeticiones, criterio de parada, regla y folds. El otro, usarHebbiano.m, permite correr una única instancia si se le dan los siguientes par\'ametros:
      
      \begin{itemize}
	\item criterioParada: es 'p' o 'o' seg\'un se quiera corte por delta pesos u ortogonalidad
	\item regla: es 's' o 'M' seg\'un se quiera usar Sanger u Oja (M)
	\item learningRate: es el valor de learningRate a usar
	\item alpha: es el valor en la f\'ormula learningRate = epoca\^(-alpha) para actualizar el learning rate. Si alpha es 0 entonces esa f\'ormula no se usa
	\item calcularPesos: es un bool que cuando es true indica que se calcule la matriz de pesos mientras que cuando es false indica que se lea una matriz ya creada
	\item cantEpocas: es la cantidad de \'epocas m\'axima que se admite (para tener otro criterio de corte)
	\item trainFilename: es el nombre (path completo) del archivo donde est\'an las instancias de entrenamiento
	\item testFilename: es el nombre (path completo) del archivo donde est\'an las instancias de validaci\'on
	\item weightsFilename: es el nombre (path completo) del archivo donde se guardar\'a (o desde donde se leer\'a si calcularPesos=false) la matriz de pesos
	\item datosFilename: es el nombre del archivo donde se escribir\'a la informaci\'on sobre la corrida
	\item figureFilename: es el nombre del .fig que se crear\'a con el gr\'afico de las instancias
      \end{itemize}
      
     \subsection{Mapeo de características}
      Al igual que en el punto anterior, main.m realiza la experimentacion detallada en el informe. Mientras que el archivo usarKohonen.m, permite correr una \'unica instancia si se le dan los siguientes par\'ametros:
      
      \begin{itemize}
	\item learningRate: es el valor de learningRate a usar
	\item sigma: es el valor que decide el rango de neuronas afectadas durante la actualizacion
	\item autoajuste: ajusta dinamicamente sigma  y learningRate a medida que anvanzan las epocas
	\item M1: es la cantidad de filas de la matriz de neuronas
	\item M2: es la cantidad de columnas de la matriz de neuronas
	\item calcularPesos: es un bool que cuando es true indica que se calcule la matriz de pesos mientras que cuando es false indica que se lea una matriz ya creada
	\item cantEpocas: es la cantidad de \'epocas m\'axima a utilizar
	\item trainFilename: es el nombre (path completo) del archivo donde est\'an las instancias de entrenamiento
	\item testFilename: es el nombre (path completo) del archivo donde est\'an las instancias de validaci\'on
	\item weightsFilename: es el nombre (path completo) del archivo donde se guardar\'a (o desde donde se leer\'a si calcularPesos=false) la matriz de pesos
	\item figureTrainFilename: es el nombre del mapa de dominantes .fig que se crear\'a con las activaciones de las instancias del archivo de entrenamiento
	\item figureTrainFilename: es el nombre del mapa de dominantes .fig que se crear\'a con las activaciones de las instancias del archivo de validaci\'on
	\item datosFilename: es el nombre del archivo donde se escribir\'a la informaci\'on sobre la corrida
      \end{itemize}
      
    
\end{document}