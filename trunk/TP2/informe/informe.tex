\documentclass[10pt,a4paper]{article}
\usepackage[utf8]{inputenc} % para poder usar tildes en archivos UTF-8
\usepackage[spanish]{babel} % para que comandos como \today den el resultado en castellano
\usepackage{a4wide} % márgenes un poco más anchos que lo usual
\usepackage{caratula}
\usepackage{geometry}

\usepackage{graphicx}
\usepackage{hyperref}
\usepackage{float}
% \usepackage{subfig}
\usepackage{caption}
\usepackage{subcaption}
\usepackage{subfiles}
\usepackage{algpseudocode}
\usepackage{placeins}


\newcommand{\myparagraph}[1]{\paragraph{#1}\mbox{}\\}

\begin{document}

\titulo{Trabajo Práctico 2}
\subtitulo{}

\fecha{11/6/2015}

\materia{Redes Neuronales Artificiales}
\grupo{}

\integrante{Landini, Federico Nicolás}{034/11}{federico91\_fnl@yahoo.com.ar}
\integrante{Rama Vilariño, Roberto Alejandro}{490/11}{bertoski@gmail.com}

\maketitle

\tableofcontents
\newpage

\subfile{introduccion.tex}
\newpage

\subfile{desarrollo.tex}
\newpage

\subfile{resultados.tex}
\newpage

\subfile{conclusiones.tex}
\newpage

\subfile{opciones.tex}
\newpage



\begin{thebibliography}{9}
  \bibitem{hertz}
    John A. Hertz, Anders S. Krogh, Richard G. Palmer,
    \emph{Introduction To The Theory Of Neural Computation},
    Westview Press,
    June 24, 1991.
  \bibitem{haykin}
    Simon O. Haykin,
    \emph{Neural Networks and Learning Machines},
    Prentice Hall,
    3rd Edition,
    November 28, 2008.
  \bibitem{nnwithJava}
    Jeff T. Heaton,
    \emph{Introduction to Neural Networks for Java},
    Heaton Research, Inc.
    November 1, 2005.
\end{thebibliography}

\end{document}
