\documentclass[informe.tex]{subfiles}
\begin{document}
  
  \section{Conclusiones}
  
  
    \subsection{Trabajo futuro}
    
      Dado el tiempo acotado para realizar pruebas con los conjuntos de datos, quedan planteados como posibles trabajos futuros el ańalisis de arquitecturas con más capas ocultas. Si bien utilizar una sola capa en general debería dar buenos resultados, es posible que usando más de una capa sea posible converger a las soluciones obtenidas en la experimentación con una menor cantidad de iteraciones o bien utilizando menos neuronas en total y en menos tiempo.
      
      Otra variante a evaluar es tratar de obtener un learning rate para cada problema que permita obtener buenos resultados m\'as r\'apidamente. Si bien se evaluaron distintos valores, es posible continuar analizando algunos otros y combinar otras arquitecturas con otros learning rates.
      
      Lo mismo podría plantearse en el problema 1 sobre el uso de momentum. En el problema 2, queda como trabajo a futuro evaluar si utilizar esta técnica permite una convergencia más rápida sin producir overfitting.

\end{document}