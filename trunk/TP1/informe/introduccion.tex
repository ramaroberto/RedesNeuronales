\documentclass[informe.tex]{subfiles}
\begin{document}
  
  \section{Introducción}

  En el presente trabajo se busca atacar dos problemas a partir del uso de redes neuronales artificiales \cite{hertz}. Si bien los dos problemas presentan características diferentes: uno consiste en diagnosticar tumores como benignos o malignos a partir de ciertas características recogidas de muestras de células en lo que puede ser catalogado como un problema de clasificación y el otro consiste en determinar los requerimientos de carga energética para calefaccionar y refrigerar edificios en función de ciertas características de los mismos, lo que puede ser catalogado como un problema de regresión; se busca evaluar el rendimiento de esta técnica y su versatilidad.
  
  Para resolver los problemas consideramos distintas configuraciones de redes para cada problema y evaluamos su rendimiento a fin de comparar los resultados y elegir la configuración que mejor se adaptaba a los conjuntos de datos pero a la vez manteniendo generalidad a fin de evitar el problema de \emph{overfitting}.
  
  Para realizar el testeo decidimos utilizar la técnica de cross-validation \cite{haykin} a fin de obtener datos más robustos respecto del rendimiento que nos permitieran tener una idea más real del nivel de generalidad de los resultados obtenidos con las redes.
  
\end{document}