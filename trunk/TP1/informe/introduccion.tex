\documentclass[informe.tex]{subfiles}
\begin{document}
  
  \section{Introducción}

  En el presente trabajo se busca atacar dos problemas a partir del uso de redes neuronales artificiales \cite{hertz}. Se nos presentaron dos problemas con características diferentes, los cuales se busca evaluar el rendimiento de esta técnica y su versatilidad. 
  
  El primero consiste en diagnosticar tumores como benignos o malignos a partir de ciertas características recogidas de muestras de células, por lo que puede ser catalogado como un problema de clasificación. El segundo consiste en determinar los requerimientos de carga energética para calefaccionar y refrigerar edificios en función de ciertas características de los mismos, lo que puede ser catalogado como un problema de regresión.
  
  Para resolver los problemas consideramos distintas configuraciones de redes para cada problema. Luego, evaluamos su rendimiento con el fin de comparar los resultados y elegir la configuración que mejor se adaptaba a los conjuntos de datos, pero a la vez manteniendo generalidad a fin de evitar el problema de \emph{overfitting}.
  
  Durante el testeo decidimos utilizar la técnica de cross-validation \cite{haykin}, con el fin de realizar un analisis mas robusto respecto al rendimiento. Esta tecnica nos permitio tener una idea más real del nivel de generalidad de los resultados obtenidos con las redes.
  
\end{document}